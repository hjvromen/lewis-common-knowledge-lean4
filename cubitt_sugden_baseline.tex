\documentclass[11pt,a4paper]{article}
\usepackage{fontspec}
\usepackage{unicode-math}
\usepackage{minted}
\usepackage{geometry}
\usepackage{parskip}
\usepackage{hyperref}
\usepackage{booktabs}
\usepackage{enumitem}

% Set fonts - Libertinus has excellent Unicode support and Times-like appearance
\setmainfont{Libertinus Serif}
\setmonofont{DejaVu Sans Mono}[Scale=0.85]
\setmathfont{Libertinus Math}

\geometry{margin=2.5cm}

% Configure hyperlinks
\hypersetup{
    colorlinks=true,
    linkcolor=blue,
    urlcolor=blue,
    citecolor=blue
}

% Configure minted for Lean code
\setminted{
    style=friendly
}
\setminted[lean4]{
    fontsize=\small,
    breaklines=true,
    linenos=false,
    bgcolor=gray!5,
    frame=single,
    framesep=2mm,
    samepage=true
}

% Tighter lists
\setlist{nosep}

\title{Cubitt Sugden Baseline}
\author{Huub Vromen}
\date{}

\begin{document}
\maketitle

\begin{minipage}{\linewidth}
\begin{minted}{lean4}
import Mathlib.Tactic
\end{minted}
\end{minipage}

\section*{Cubitt \& Sugden's Syntactic Approach to Lewis's Theory of Common Knowledge}


\subsection*{Overview}


Cubitt and Sugden (2003) reconstructed Lewis's theory using a syntactic approach where "reason to believe" and "indication" are primitive relations.


This file formalizes their approach and extends it by proving Lewis's theorem for the \textbf{complete infinite hierarchy}, whereas Cubitt and Sugden prove only the first few levels.


\subsubsection*{Conceptual Framework}


\begin{itemize}
  \item Treats \textbf{reason to believe} as a primitive relation \texttt{R : indiv → Prop → Prop}.
  \item Treats \textbf{indication} as a primitive relation \texttt{Ind : Prop → indiv → Prop → Prop}.
  \item Based on the idea that each agent has a "reasoning scheme" (logic + axioms).
\end{itemize}

The key insight is that common knowledge emerges from \textbf{reflexive common indicators}: states of affairs that indicate both a target proposition and their own recognition by all agents.


\subsubsection*{Key Features}


\paragraph{Strengths}

\begin{itemize}
  \item Proves Lewis's theorem correctly for the first few steps of the infinite hierarchy.
  \item Makes all assumptions explicit as axioms.
  \item Provides clear axiomatic foundation.
\end{itemize}

\paragraph{Limitations (Addressed by Vromen 2024)}

\begin{itemize}
  \item \texttt{A1} and \texttt{A6} must be assumed as axioms.
  \item Implies logical omniscience (agents have reason to believe all tautologies).
  \item Cannot represent conflicting reasons to believe.
  \item The "thereby" in Lewis's definition of indication is not formally captured.
\end{itemize}

\subsubsection*{Comparison with Other Approaches}


\begin{center}
\begin{tabular}{lll}
\hline
Feature & Cubitt-Sugden (this file) & Vromen (Justification Logic) \\
\hline
\textbf{R operator} & Primitive relation \texttt{R} & ∃ evidence \texttt{t}, \texttt{t:F} \\
\textbf{Indication} & Primitive relation \texttt{Ind} & \texttt{R i (A → φ)} \\
\textbf{A1 status} & \textbf{AXIOM} & \textbf{THEOREM} $\checkmark$ \\
\textbf{A6 status} & \textbf{AXIOM} & \textbf{THEOREM} $\checkmark$ \\
\textbf{Logical omniscience} & Yes & No $\checkmark$ \\
\hline
\end{tabular}
\end{center}


\subsubsection*{References}


\begin{itemize}
  \item \textbf{Cubitt, R. P., \& Sugden, R. (2003).} Common knowledge, salience and convention: A reconstruction of David Lewis’ game theory. \textit{Economics and Philosophy, 19}, 175-210.
  \item \textbf{Lewis, D. (1969).} \textit{Convention: A Philosophical Study.} Cambridge, MA: Harvard University Press.
  \item \textbf{Vromen, H. (2024).} Reasoning with reasons: Lewis on common knowledge. \textit{Economics \& Philosophy, 40}(2), 397–418.
\end{itemize}

\subsubsection*{File Structure}


\begin{enumerate}
  \item  Type parameters and primitive operators
  \item  Axioms \texttt{A1} and \texttt{A6}
  \item  Conditions \texttt{C1}–\texttt{C4} for reflexive common indicators
  \item  Lemmas \texttt{L1}–\texttt{L4} demonstrating first few levels
  \item  General theorem via R-closure induction
\end{enumerate}


\begin{minipage}{\linewidth}
\begin{minted}{lean4}
section Lewis
\end{minted}
\end{minipage}

---

\subsection*{Section 1: Type Parameters and Primitive Operators}


We introduce the basic types and the two primitive relations that form the foundation of Cubitt and Sugden's approach.


\subsubsection*{Type Parameters}


\begin{itemize}
  \item \texttt{indiv}: The type of individuals (agents) in the population.
  \item \texttt{A}: A state of affairs that may or may not hold.
  \item \texttt{φ}: A proposition that agents may have reason to believe.
\end{itemize}


\begin{minipage}{\linewidth}
\begin{minted}{lean4}
variable {indiv : Type*}
variable (A φ : Prop)
\end{minted}
\end{minipage}

\subsubsection*{Primitive Operators}


\paragraph{1. Reason to Believe (\texttt{R})}


\begin{quote}
\textbf{\texttt{R i p}} — "Individual \texttt{i} has reason to believe proposition \texttt{p}"
\end{quote}


\textbf{Interpretation:} "To say that some individual \texttt{i} has reason to believe some proposition \texttt{x} is to say that \texttt{x} is true within some logic of reasoning that is \textit{endorsed} by (that is, accepted as a normative standard by) person \texttt{i}" (Cubitt \& Sugden 2003, p. 184).


\begin{itemize}
  \item \textbf{Status:} Primitive relation (undefined, taken as basic).
  \item \textbf{Nature:} This is a normative notion: it captures what \texttt{i} \textit{should} believe given their reasoning standards, not what they actually believe.
\end{itemize}

\paragraph{2. Indication (\texttt{Ind})}


\begin{quote}
\textbf{\texttt{Ind A i p}} — "State of affairs \texttt{A} indicates to \texttt{i} that \texttt{p}"
\end{quote}


Lewis (1969, p. 52–53) defines indication counterfactually: "\textit{A indicates} to someone \textit{x} that \_\_\_ if and only if, if \textit{x} had reason to believe that \textit{A} held, \textit{x} would \textit{thereby} have reason to believe that \_\_\_."


The word "thereby" suggests the reason for \_\_\_ depends on the reason for \texttt{A}, a dependence that this primitive treatment cannot capture internally.


\textbf{Interpretation} "In the logic of reasoning that \texttt{i} endorses, there is an inference rule which legitimates inferring \texttt{x} from '\texttt{A} holds'." (Cubitt \& Sugden 2003, p. 187).


\begin{itemize}
  \item \textbf{Status:} Primitive relation (undefined, taken as basic).
  \item \textbf{Limitation:} The "thereby" in Lewis's definition of indication is not formally captured.
\end{itemize}


\begin{minipage}{\linewidth}
\begin{minted}{lean4}
variable {R : indiv → Prop → Prop}
variable {Ind : Prop → indiv → Prop → Prop}
\end{minted}
\end{minipage}

---

\subsection*{Section 2: Axioms}


Two axioms govern the interaction between \texttt{R} and \texttt{Ind}. These must be assumed because the operators are primitive—we cannot derive properties from their structure.


\subsubsection*{Axiom \texttt{A1}: Principle of detachment}


If \texttt{A} indicates \texttt{p} to \texttt{i}, and \texttt{i} has reason to believe \texttt{A}, then \texttt{i} has reason to believe \texttt{p}.


\begin{quote}
\textbf{Formula:} \texttt{Ind A i p → R i A → R i p}
\end{quote}


This is Lewis's "principle of detachment," used at every step of his argument. Without \texttt{A1}, the proof cannot proceed.


\textbf{Why it's an axiom:} Since \texttt{Ind} is primitive, we cannot derive this from its definition. Vromen's approach defines \texttt{Ind A i p := R i (A → p)}, making \texttt{A1} provable from the application rule of justification logic.


\begin{minipage}{\linewidth}
\begin{minted}{lean4}
variable (A1 : ∀ {i : indiv} {p : Prop}, Ind A i p → R i A → R i p)
\end{minted}
\end{minipage}

\subsubsection*{Axiom \texttt{A6}: Transitivity Through Shared Standards}


If \texttt{A} indicates to \texttt{i} that \texttt{j} has reason to believe \texttt{A}, and \texttt{i} has reason to believe that \texttt{A} indicates \texttt{p} to \texttt{j}, then \texttt{A} indicates to \texttt{i} that \texttt{j} has reason to believe \texttt{p}.


\begin{quote}
\textbf{Formula:} \texttt{Ind A i (R j A) ∧ R i (Ind A j p) → Ind A i (R j p)}
\end{quote}


This axiom expresses that agents can reason about each other's reasoning under shared inductive standards. It formalizes Lewis's informal appeal to "ancillary premises regarding our rationality, inductive standards, and background information."


\textbf{Why it's an axiom:} Like \texttt{A1}, this cannot be derived from primitive operators. Vromen proves it as a theorem from the compositional structure of reason terms.


\begin{minipage}{\linewidth}
\begin{minted}{lean4}
variable (A6 : ∀ {i j : indiv} {u : Prop}, Ind A i (R j A) ∧ R i (Ind A j u) → Ind A i (R j u))
\end{minted}
\end{minipage}

---

\subsection*{Section 3: Conditions for Reflexive Common Indicators}


A state of affairs \texttt{A} is a \textit{reflexive common indicator} in population \texttt{P} that \texttt{φ} if these four conditions hold. Together with axioms \texttt{A1} and \texttt{A6}, they suffice to prove Lewis's theorem.


\subsubsection*{Condition \texttt{C1}: Publicity}


Everyone has reason to believe \texttt{A} holds. This is the foundation: there must be some publicly recognized state of affairs.


\textbf{Examples}: A public announcement, a salient environmental feature (Lewis's candle on the table), or a shared observation (lights going out).


\begin{minipage}{\linewidth}
\begin{minted}{lean4}
variable (C1 : ∀ i : indiv, R i A)
\end{minted}
\end{minipage}

\subsubsection*{Condition \texttt{C2}: Reflexivity}


\texttt{A} indicates to each agent that every other agent has reason to believe \texttt{A}. This is the "reflexive" aspect.


\begin{minipage}{\linewidth}
\begin{minted}{lean4}
variable (C2 : ∀ i j : indiv, Ind A i (R j A))
\end{minted}
\end{minipage}

\subsubsection*{Condition \texttt{C3}: Target Indication}


\texttt{A} indicates \texttt{φ} to every agent. This connects the state of affairs to the target proposition that will become common knowledge.


\begin{minipage}{\linewidth}
\begin{minted}{lean4}
variable (C3 : ∀ i : indiv, Ind A i φ)
\end{minted}
\end{minipage}

\subsubsection*{Condition \texttt{C4}: Shared Inductive Standards}


If \texttt{A} indicates \texttt{u} to \texttt{i}, then \texttt{i} has reason to believe that \texttt{A} indicates \texttt{u} to \texttt{j}.


\begin{quote}
\textbf{Formula:} \texttt{Ind A i u → R i (Ind A j u)}
\end{quote}


This formalizes shared reasoning standards—agents know they reason alike. This formalizes Lewis' assumption about "common inductive standards and background information". It's what allows \texttt{i} to reason about what \texttt{j} can infer from \texttt{A}.


As Cubitt \& Sugden note (p. 189), this is actually stronger than strictly necessary - we only need it for certain types of propositions. But it simplifies the formalization.


\begin{minipage}{\linewidth}
\begin{minted}{lean4}
variable (C4 : ∀ {i j : indiv} {u : Prop}, Ind A i u → R i (Ind A j u))

include A1 A6 C1 C2 C3 C4
\end{minted}
\end{minipage}

---

\subsection*{Section 4: The Iteration of Reasons to Believe}


We now prove that conditions \texttt{C1}–\texttt{C4} together with axioms \texttt{A1} and \texttt{A6} generate an infinite hierarchy of nested reasons to believe. We start with concrete lemmas for the first few levels, then prove the general theorem.


\textbf{Pattern:}

\begin{itemize}
  \item \texttt{L1}: \texttt{R i φ} (first-order)
  \item \texttt{L2}: \texttt{R i (R j φ)} (second-order)
  \item \texttt{L3}: \texttt{R i (R j (R k φ))} (third-order)
  \item \texttt{L4}: \texttt{R i (R j (R k (R l φ)))} (fourth-order)
  \item ...and so on to infinity
\end{itemize}


\subsubsection*{Lemma \texttt{L1}: First-Order Reasons to Believe}


Every individual has reason to believe \texttt{φ}.


\textbf{Proof:} By \texttt{C1}, \texttt{i} has reason to believe \texttt{A}. By \texttt{C3}, \texttt{A} indicates \texttt{φ} to \texttt{i}. By \texttt{A1}, \texttt{i} has reason to believe \texttt{φ}.


\begin{minipage}{\linewidth}
\begin{minted}{lean4}
omit A6 C2 C4 in
lemma L1 (i : indiv) : R i φ := by
  have h₁ : R i A := C1 i
  have h₂ : Ind A i φ := C3 i
  exact A1 h₂ h₁
\end{minted}
\end{minipage}

Alternative proof of \texttt{L1} using automation:


\begin{minipage}{\linewidth}
\begin{minted}{lean4}
omit A6 C2 C4 in
lemma L1_aesop (i : indiv) : R i φ := by aesop
\end{minted}
\end{minipage}

\subsubsection*{Lemma \texttt{L2}: Second-Order Reasons to Believe}


Every individual \texttt{i} has reason to believe that every individual \texttt{j} has reason to believe \texttt{φ}.


\textbf{Proof outline:}

\begin{enumerate}
  \item  \texttt{A} indicates \texttt{φ} to \texttt{i} (\texttt{C3})
  \item  \texttt{i} knows \texttt{j} shares inductive standards (\texttt{C4})
  \item  \texttt{A} indicates to \texttt{i} that \texttt{j} can infer \texttt{φ} from \texttt{A} (\texttt{C2} + \texttt{A6})
  \item  \texttt{i} has reason to believe \texttt{A}, so \texttt{i} has reason to believe \texttt{R j φ} (\texttt{A1})
\end{enumerate}


\begin{minipage}{\linewidth}
\begin{minted}{lean4}
lemma L2 (i j : indiv) : R i (R j φ) := by
  have h₁ : Ind A i φ := C3 i
  have h₂ : R i (Ind A j φ) := C4 h₁
  have h₃ : Ind A i (R j φ) := A6 ⟨C2 i j, h₂⟩
  exact A1 h₃ (C1 i)
\end{minted}
\end{minipage}

\textbf{Observation on automation:} The automated proof (\texttt{L2\_aesop}) succeeds WITHOUT \texttt{A6} or \texttt{C4}. This is surprising and reveals something important:


At depth 2, the logical structure is simpler. Automation can find shortcuts that don't need the full machinery. But at depth 3 and beyond, \texttt{A6} and \texttt{C4} become essential - the shortcuts don't work anymore.


\textbf{Philosophical lesson:} The second level of "common reason to believe" is logically simpler than higher levels. This might explain why everyday reasoning often stops at "I know that you know" without going deeper.


\begin{minipage}{\linewidth}
\begin{minted}{lean4}
omit A6 C4 in
lemma L2_aesop (i j : indiv) : R i (R j φ) := by aesop
\end{minted}
\end{minipage}

\subsubsection*{Lemma \texttt{L3}: Third-Order Reasons to Believe}


Every individual \texttt{i} has reason to believe that every individual \texttt{j} has reason to believe that every individual \texttt{k} has reason to believe \texttt{φ}.


\begin{minipage}{\linewidth}
\begin{minted}{lean4}
lemma L3 (i j k : indiv) : R i (R j (R k φ)) := by
  have h₁ : Ind A i φ := C3 i
  have h₂ : R i (Ind A k φ) := C4 h₁
  have h₃ : Ind A i (R k φ) := A6 ⟨C2 i k, h₂⟩
  have h₄ : R i (Ind A j (R k φ)) := C4 h₃
  have h₅ : Ind A i (R j (R k φ)) := A6 ⟨C2 i j, h₄⟩
  exact A1 h₅ (C1 i)

lemma L3_aesop (i j k : indiv) : R i (R j (R k φ)) := by aesop
\end{minted}
\end{minipage}

\subsubsection*{Lemma \texttt{L4}: Fourth-Order Reason to Believe}


\begin{minipage}{\linewidth}
\begin{minted}{lean4}
lemma L4 (i j k l : indiv) : R i (R j (R k (R l φ))) := by
  have h₁ : Ind A i φ := C3 i
  have h₂ : R i (Ind A l φ) := C4 h₁
  have h₃ : Ind A i (R l φ) := A6 ⟨C2 i l, h₂⟩
  have h₄ : R i (Ind A k (R l φ)) := C4 h₃
  have h₅ : Ind A i (R k (R l φ)) := A6 ⟨C2 i k, h₄⟩

  have h₆ : R i (Ind A j (R k (R l φ))) := C4 h₅
  have h₇ : Ind A i (R j (R k (R l φ))) := A6 ⟨C2 i j, h₆⟩
  exact A1 h₇ (C1 i)

lemma L4_aesop (i j k l : indiv) : R i (R j (R k (R l φ))) := by aesop
\end{minted}
\end{minipage}

---

\subsection*{Section 5: The General Theorem}


Cubitt \& Sugden stop halfway \texttt{L4} and say "And so on", suggesting that the pattern continues indefinitely. This will be proved as follows.


Instead of proving each level separately, we can define the \textbf{R-closure} inductively and prove that for ANY proposition in this closure, everyone has reason to believe it.


\subsubsection*{Definition: R-Closure (\texttt{RC})}


The R-closure of \texttt{φ} is the smallest set of propositions that:

\begin{enumerate}
  \item  Contains \texttt{φ} itself (base case)
  \item  Is closed under "j has reason to believe" for all \texttt{j} (inductive case)
\end{enumerate}

\begin{quote}
\texttt{RC φ} contains: \texttt{φ}, \texttt{R j φ}, \texttt{R i (R j φ)}, \texttt{R k (R i (R j φ))}, etc.
\end{quote}


\begin{minipage}{\linewidth}
\begin{minted}{lean4}
inductive RC (φ : Prop) : Prop → Prop where
  | base : RC φ φ
  | step {u : Prop} (j : indiv) (hu : RC φ u) : RC φ (R j u)
\end{minted}
\end{minipage}

\subsubsection*{Key Lemma: Indication Propagates Through R-Closure}


If \texttt{q} is in the R-closure of \texttt{φ}, then \texttt{A} indicates \texttt{q} to every individual.


This is the heart of Lewis's theorem. It shows that conditions \texttt{C1}–\texttt{C4} make \texttt{A} indicate not just \texttt{φ}, but all its higher-order iterations.


\textbf{Proof by structural induction}:

\begin{itemize}
  \item \textbf{Base case:} \texttt{A} indicates \texttt{φ} to \texttt{i} by \texttt{C3}.
  \item \textbf{Inductive step:} If \texttt{A} indicates \texttt{u} to \texttt{i}, then by \texttt{C4} we get \texttt{R i (Ind A j u)}, and by \texttt{A6} we get \texttt{Ind A i (R j u)}.
\end{itemize}


\begin{minipage}{\linewidth}
\begin{minted}{lean4}
omit A1 C1 in
lemma rc_implies_indication {i : indiv} {q : Prop} (h : @RC indiv R φ q) : Ind A i q := by
  induction h with
  | base => exact C3 i
  | @step v j _ ih =>
      have h₁ : R i (Ind A j v) := C4 ih
      have h₂ : Ind A i (R j v) := A6 ⟨C2 i j, h₁⟩
      exact h₂
\end{minted}
\end{minipage}

\subsubsection*{Lewis's Theorem: Common Reason to Believe}


For any proposition \texttt{p} in the R-closure of \texttt{φ}, every individual has reason to believe \texttt{p}.


This is the formalization of Cubitt \& Sugden's "Lewis' Theorem": "For all states of affairs \texttt{A}, for all propositions \texttt{x}, and for all populations \texttt{P}: if \texttt{A} holds, and if \texttt{A} is a reflexive common indicator in \texttt{P} that \texttt{x}, then \texttt{r\textasciicircum{}P(x)} is true."


\begin{description}
  \item[Significance] Common knowledge is not an infinite regress (circular) but an infinite \textit{consequence}of finite, non-circular conditions.
\end{description}


\begin{minipage}{\linewidth}
\begin{minted}{lean4}
lemma everyone_reason_of_rc (hp : @RC indiv R φ p) : ∀ i : indiv, R i p := by
  intro i
  have hInd : Ind A i p := rc_implies_indication A φ A6 C2 C3 C4 hp
  exact A1 hInd (C1 i)

end Lewis
\end{minted}
\end{minipage}

---

\subsection*{Section 6: Summary and Assessment}


\subsubsection*{Philosophical Significance}


This formalization vindicates Lewis's claim that his account avoids standard objections to common knowledge:


\begin{enumerate}
  \item  \textbf{No unbounded reasoning}: The proof uses only two inductive steps.
  \item  \textbf{No mind-reading}: Reasoning concerns public \texttt{A}, not private mental states.
  \item  \textbf{Explains genesis}: Shows how common knowledge arises, not just what it is.
\end{enumerate}

\subsubsection*{Achievements}


\begin{itemize}
  \item $\checkmark$ Lewis's theorem proven for the complete infinite hierarchy.
  \item $\checkmark$ All assumptions made explicit as axioms.
  \item $\checkmark$ Concrete examples (\texttt{L1}–\texttt{L4}) demonstrate the pattern.
\end{itemize}

\subsubsection*{Limitations}


\begin{itemize}
  \item $\times$ \texttt{A1} and \texttt{A6} assumed without explanation.
  \item $\times$ Logical omniscience (agents have reason to believe all tautologies).
  \item $\times$ Cannot represent conflicting reasons.
  \item $\times$ "Thereby" in indication not formally captured.
\end{itemize}

\subsubsection*{Vromen's Improvements}


Vromen (2024) addresses these limitations using justification logic:


\begin{itemize}
  \item Defines \texttt{R} as \texttt{∃r. rb r i φ} (existential over explicit reasons).
  \item Defines \texttt{Ind} as \texttt{R i (A → φ)} (reason for the implication).
  \item Proves \texttt{A1} and \texttt{A6} as theorems from more primitive principles.
  \item Avoids logical omniscience (only three tautologies needed).
  \item Can represent conflicting reasons.
\end{itemize}

\subsubsection*{Recommendation}


\begin{itemize}
  \item For the \textit{structure} of Lewis's argument → \texttt{this file}.
  \item For the \textit{foundations} of Lewis's theory → \texttt{Vromen\_justification\_logic.lean}.
  \item For what \textit{doesn't work} → \texttt{Sillari\_refutation.lean}.
\end{itemize}


\end{document}